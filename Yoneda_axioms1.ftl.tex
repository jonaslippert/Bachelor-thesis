\documentclass{article}

\title{The Yoneda Lemma}
\author{Jonas Lippert}
\date{2021}

\usepackage{tikz-cd}
\usepackage[utf8]{inputenc}
\usepackage[english]{babel}
\usepackage{naproche}
\usepackage{bbm}
\usepackage{hyperref}


\newcommand{\mcirc}{\circ_{c}}
\newcommand{\inn}{~\epsilon~}
\newcommand{\innn}{~\epsilon~} % "... is an arrow and \inn ..."


\newcommand{\Naproche}{$\mathbbm{N}$aproche}


\begin{document}
	
	\maketitle
	
	
	
	\section{Introduction}
	
	The Yoneda-Lemma is an early result of category theory. Following the book by Emily Riehl we want to formalize it taking an arrow theoretic approach.
	\newpage
	
	\section{Categories}
	\begin{comment}
	\begin{forthel}
	[synonym arrow/-s][synonym category/categories][synonym functor/-s]
	
	[checktime 10]
	[timelimit 10]
	[depthlimit 10]
	[checkdepth 10]
	\end{forthel}
	\end{comment}
	
	\begin{forthel}
		
		\begin{signature} An arrow is a notion. \end{signature}
		\begin{signature} A collection of arrows is a notion. \end{signature}
		
		Let $f,g,h$ denote arrows.
		Let $C,D$ denote collection of arrows.
		
		\begin{signature} $f\inn C$ is an atom. \end{signature}
		
		\begin{axiom}
			$C = D \iff (f\inn C \iff f \inn D)$.
		\end{axiom}
		
		\begin{signature} $s[f]$ is an arrow. \end{signature}   
		\begin{signature} $t[f]$ is an arrow. \end{signature}  
		
		\begin{signature} $g \mcirc f$ is an arrow.\end{signature}
		
		
		\begin{definition} 
			A category is a collection of arrows $C$ such that
			
			(for every arrow $f$ such that $f\inn C$ we have
			
			$s[f] \inn C$ and $t[f] \inn C$ and
			$t[s[f]]=s[f]$ and $s[t[f]]=t[f]$ and
			 
			\begin{center}
				\begin{tikzcd} 
					s[f] \ar{r}{s[f]} \ar{rd}[swap]{f} & s[f] \ar{d}{f} \\ & t[f]
				\end{tikzcd}
			\end{center}	
			and
			\begin{center}
				\begin{tikzcd} 
					s[f] \ar{r}{f} \ar{rd}[swap]{f} & t[f] \ar{d}{t[f]} \\ & t[f]
				\end{tikzcd}
				)
			\end{center}
			and (for each arrow $f,g$ such that $f,g \inn C$ we have
			
			($t[f]=s[g] \implies$ (there is an arrow $h$ such that $h \inn C$ and 

			\begin{center}
				\begin{tikzcd} 
					s[f] \ar{r}{f} \ar{rd}[swap]{h} & s[f] \ar{d}{g} \\ & t[g]
				\end{tikzcd}
			\end{center}						
			
			and for every arrow $k$ such that $k \inn C$ and $g \mcirc f = k$ we have $h=k$)))
			
			and for all arrows $f,g,h$ such that $f,g,h \inn C$ and $t[f]=s[g]$ and $t[g]=s[h]$
			\begin{center}
				\begin{tikzcd} 
					s[f] \ar{r}{f} \ar{d}[swap]{(g \mcirc f)} & t[f] \ar{d}{(h \mcirc g)} \\ t[g] \ar{r}{h} & t[h]
				\end{tikzcd}	.
			\end{center}
		                      		 
		\end{definition}
		

	\end{forthel}
	
	\section{Construction of SET}
	
	\begin{comment}
	\begin{forthel}
	
	Let $f \innn C$ stand for ($f$ is an arrow such that $f \inn C$).
	
	\end{forthel}
	\end{comment}
	
	\begin{forthel}
		

		\begin{definition} 
			An sset is a set $x$ such that $x$ is an element of some set.
		\end{definition}
		
		Let $f,g,h$ denote functions.
		
		\begin{signature} 
			$Cod(f)$ is a notion. 
		\end{signature}

		\begin{axiom} 
			$Cod(f)$ is a set. 
		\end{axiom}
		
		\begin{axiom} 
			Let $x \in  Dom(f)$. $f(x) \in Cod(f)$.
		\end{axiom}

		\begin{axiom}[Ext]
			Let $f,g$ be functions and $Dom(f) = Dom(g)$ and $Cod(f) = Cod(g)$.
			Let $f(x) = g(x)$ for every element $x$ of $Dom(f)$.
			$f = g$.
		\end{axiom}
		
		\begin{definition} 
			Let $Cod(f)=Dom(g)$. 
			
			$g \circ f$ is the function $h$ such that
			$Dom(h)=Dom(f)$ and $Cod(h)=Cod(g)$ and $h(x)=g(f(x))$ for every element $x$ of $Dom(f)$.
		\end{definition}
		

		
		\begin{axiom}
			Every function is an arrow.
		\end{axiom}
		
		\begin{axiom}		
			$s[f]$ is a function such that 
			
			$Dom(s[f])=Dom(f)=Cod(s[f])$.
		\end{axiom}
		
		\begin{axiom} 
			$s[f](y)=y$ for every element $y$ of $Dom(f)$.
		\end{axiom}
		

		\begin{axiom}		
			$t[f]$ is a function such that 
			
			$Dom(t[f])=Cod(f)=Cod(t[f])$.
		\end{axiom}
		
		\begin{axiom} 
			$t[f](y)=y$ for every element $y$ of $Cod(f)$.
		\end{axiom}
		
		\begin{definition} 
		\begin{center}
			$SET = \{ $function $f \mid Dom(f)$ is an sset and $Cod(f)$ is an sset $\}$.
		\end{center}
			
		\end{definition}
		
		\begin{axiom} 
			$SET$ is a set.
		\end{axiom}
		
		\begin{axiom} 
			If $Dom(f), Cod(f)$ are sset then $f$ is setsized.
		\end{axiom}
		

		

		
		\begin{axiom}
			$SET$ is a collection of arrows.
		\end{axiom}


		
		\begin{axiom}
			Let $f$ be an arrow. $f \inn SET \iff f \in SET$.
		\end{axiom}
		
		\begin{axiom}
		  Let $f \innn SET$. $s[f]=Dom(f)$ and $t[f]=Cod(f)$.
		\end{axiom}
		
		\begin{axiom} 
			Let $f,g \in SET$ and $Cod(f)=Dom(g)$. $g \circ f = g \mcirc f$.
		\end{axiom}

	
		


	\end{forthel}
	
		
	\section{Bijections}
	
	\begin{forthel}
		
		\begin{signature}
			Let $Q,R$ be sets.
			A bijection between $Q$ and $R$ is a notion.
		\end{signature}

		\begin{axiom}
			Let $Q,R$ be sets.
			Let $f$ be a function such that $Dom(f) = Q$ and $Cod(f)=R$.
			Let $g$ be a function such that $Dom(g) = R$ and $g(y) \in Q$ for any element $y$ of $R$.
			Let $f(g(y))=y$ for all elements $y$ of $Dom(g)$. 
			Let $g(f(x))=x$ for all elements $x$ of $Dom(f)$.
			Then $f$ is a bijection between $Q$ and $R$.
		\end{axiom}
		
	\end{forthel}
	
	\section{Functors}
	
	\begin{forthel}
		
		\begin{signature} A functor is a notion.
		\end{signature}
		
		\begin{signature} Let $F$ be a functor. Let $f$ be an arrow. $F[f]$ is an arrow.
		\end{signature}
		
		\begin{definition}
			Let $C,D$ be categories.
			A functor from $C$ to $D$ is a functor $F$ such that
			(for all arrows $f$ such that $f \inn C$ we have
			$F[f] \inn D$
			and $$F[s[f]] = s[F[f]]$$
			and $$F[t[f]] = t[F[f]])$$
			and for all arrows $f,g$ such that $f,g \inn C$ and $t[f]=s[g]$ we have
			\begin{center}
				\begin{tikzcd} 
					F[s[f]] \ar{r}{F[f]} \ar{rd}[swap]{F[g \mcirc f]} & F[t[f]] \ar{d}{F[g]} \\  & F[t[g]]
				\end{tikzcd}.
			\end{center}
		\end{definition}
		
	\end{forthel}
	
	\section{Construction of the Hom Functor}
	

	
	\begin{forthel}
		
		Let $C$ denote a category.	
		
		\begin{signature}
			Let $c,x \innn C$. 
			$Hom[C,c,x]$ is a collection of arrows such that $f \inn C$ for any arrow $f$ such that $f \inn Hom[C,c,x]$.
		\end{signature}
		
		\begin{axiom} 
			Let $c,x \innn C$. Let $h$ be an arrow. 
			\begin{center}
				$h \inn Hom[C,c,x]$ $\iff (s[h]=c$ and $t[h]=x$).
			\end{center}
		\end{axiom} 
				
		\begin{definition}
			A locally small category is a category $C$ such that 
			$Hom[C,c,f]$ is an element of SET for all arrows $c,f$ such that $c,f \inn C$.
		\end{definition}
		
		
		Let $C$ denote a locally small category.
		
		\begin{axiom}
			Let $c,x \innn C$. Let $h$ be an arrow. 
			$$h \in Dom(Hom[C,c,x]) \iff h \inn Hom[C,c,x].$$
		\end{axiom}

		
		\begin{axiom}
			Let $c,f \innn C$.
			$$Dom(Hom[C,c,f]) = Hom[C,c,s[f]]$$ and 
			$$Cod(Hom[C,c,f]) = Hom[C,c,t[f]]$$ and
			$$Hom[C,c,f](h) = f \mcirc h$$ for each arrow $h$ such that $h \in Hom[C,c,s[f]]$.
		\end{axiom}

		\begin{axiom} 
			Let $c,x \innn C$. Any element $h$ of $Dom(Hom[C,c,x])$ is an arrow.
		\end{axiom}
		

		

		
		\begin{definition} 
			Let $c \innn C$.
			$HomF[C,c]$ is a functor such that 
			for each arrow $f$ such that $f \inn C$ we have $$HomF[C,c][f] = Hom[C,c,f].$$
		\end{definition}
		
		

  \begin{lemma}
  Contradiction.
  \end{lemma}
		


	\end{forthel}
	
\end{document}
